\newcommand{\CLASSINPUTtoptextmargin}{1.9cm}
\newcommand{\CLASSINPUTbottomtextmargin}{1.9cm}
\documentclass[conference,a4paper]{IEEEtran}

\usepackage[hyphens]{url}
\usepackage[hidelinks]{hyperref}
\usepackage{cite}
\usepackage[super]{nth}
\usepackage{graphicx}

% Plot graphics
\usepackage{pgfplots}
\usepackage{etoolbox}
\usepgfplotslibrary{dateplot}
\pgfplotsset{compat=1.12}

\usepackage{todonotes}
\usepackage{kantlipsum}


\begin{document}

% ----------------------
%    Document header
% ----------------------
\title{%
  \textit{Predator-drone}, one drone to rule them all%
}
\author{%
  Florent Fayollas and Antoine Vacher,\\%
  \textit{Masters students at TLS-SEC}%
}
\maketitle


\begin{abstract}
  Drones are now present almost everywhere. Reserved before to military, this technology
  is now accessible to everyone. Their feild of use is wide: surveillance, agriculture,
  media coverage, etc. However, some people are using them in malicious way, to spy,
  to bomb, etc. In this work, we developed a tool whose goal is to take control of
  multiples civil drones.
\end{abstract}
\vspace*{1.5em}

\begin{IEEEkeywords}
  Security, UAV, Drone, Hijack, Parrot AR.Drone 2.0, Syma X5C-1, WiFi, RF 2.4 GHz, Deauth attack
\end{IEEEkeywords}
\vspace*{1.5em}


% ----------------------
%    Document content
% ----------------------

\section{Introduction}
\subsection{Overall introduction}
Unmanned aerial vehicles (UAV), commonly known as drones, are aircrafts without human
pilot aboard. An UAV is a component of an unmanned aircraft system (UAS), which include an
UAV, a ground-based controller and a communication system between the two. However, many
UAVs can take certain decisions autonomously during their flight.

These systems were originally developed by military and used in too dangerous missions for
humans. In the past few years, their use became generalized to many sectors such as
academic, commercial, and even recreational. As concrete examples, drones are now used for
surveillance, agriculture and aerial photography.

Their field of use is continuously growing. As an example, Amazon is working on drones to
be used for their deliveries: the future Amazon Prime Air service. Amazon managed to
perform its first fully autonomous delivery on December 7, 2016~\cite{bib:amazon}.

\subsection{Motivation of this work}
The use cases stated before are mostly advantageous for our society. Yet, malicious usages
also exists. A first concrete example is the recreational drone use by Daesh to bomb on Syrian
frontline~\cite{bib:daesh}. Another one, more recent, is the flight over London Gatwick
airport by a non-identified drone, which caused a blockage of the whole
airport~\cite{bib:gatwick}.

Thereby, it is necessary to protect from these threats. Military has developed solutions,
such as control link jamming. The DroneGun, constructed by
DroneShield~\cite{bib:droneshield}, is a great example. Nevertheless, such solutions
do not exist for civilians.

We could think on making military solutions accessible publicly, but these solutions often
take down the drone, without considering the final state of the drone. This means that
these solutions can cause a crash. This is not acceptable for civilians. As an example, we
could think of a drone flighting over a crowd, that a crash would harm.

\subsection{Our goals}
The objective of this work is to develop a tool able to take control of multiple
commercial drones, without implying their crash. This tool will be embedded on a predator
drone that will cover a restricted flying zone.

In ou study, we focused on two commercial drones: the Parrot AR.Drone 2.0 and the Syma
X5C-1. Then, we addressed the embedding question on a predator drone.

\section{Hijacking a Parrot AR.Drone}
\section{Hijacking a Syma X5C-1}
\section{Embedding tool on a predator drone}
\section{Conclusion and future applications}



%\ref{fig:onlyoffice}
%\begin{figure}[!hb]
  %\centering
  %\includegraphics[width=\linewidth]{}
  %\caption{}%
  %\label{fig:}
%\end{figure}



% ------------------
%    Bibliography
% ------------------

\begin{thebibliography}{1}
  \bibitem{bib:amazon}
    \url{https://www.amazon.com/Amazon-Prime-Air/}

  \bibitem{bib:daesh}
    \url{https://ctc.usma.edu/islamic-state-drones-supply-scale-future-threats/}

  \bibitem{bib:gatwick}
    \url{https://www.bbc.com/news/uk-england-sussex-46623754}

  \bibitem{bib:droneshield}
    \url{https://www.droneshield.com/}

\end{thebibliography}

\end{document}
